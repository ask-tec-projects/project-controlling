\chapter{De 7 "Themes"}

\section{Business Case}
\label{sec:business_case}
Business Case er idéen, der bygges et projekt op omkring. Idéen skal repræsentere potentiel værdi for organisationen. Temaet anvendes til løbende at vurdere om denne værdi eksisterer og er realistisk at udarbejde, for at retfærdiggøre initielle og yderligere investeringer. Vurderes en business case til ikke længere at levere den nødvendige værdi bør projektet stoppes.

\section{Organization}
Et af de 7 PRINCE2 principper omhandler en veldefineret organisation og PM struktur. Temaet uddyber strukturen ved følgende retningslinjer angående hvordan et PM team bør opsættes:
\begin{itemize}
    \item Indeholde repræsentanter fra kunde, organisation og supplier
    \item Veldefinerede og afgrænsede ansvarsområder for de inkluderede roller
    \item Løbende review af projektroller
    \item En veldokumenteret kommunikationsstrategi
\end{itemize}
For at kunne fungere i et stort sæt af miljøer fremskriver PRINCE2 ikke rigide management funktioner, men udlægger istedet organisations roller, der er associeret med bestemte ansvarsområder indenfor projektet. Det gør at en organisation lettere kan omstille sig til PRINCE2 og integrere det med eksisterende business processer.

\section{Quality}
Målet med Quality temaet er at definere og implementere metoder til at validere kvaliteten af projekter. Heri er kernen at sikre at både organisationens interne forventninger mødes samt at den endelige kunde opnår den ønskede effekt af projektet.

Under begge disse mål er det vigtigt at holde projektets scope for øje. Skrider scopet udover det nødvendige beskrevet i Project Description kan de have en negativ indflydelse på organisationens interne forventninger, da der kan være brugt flere resourcer en højst nødvendigt. På den anden side, hvis scope ikke er fuldt mødt, kan det have negative konsekvenser for den endelige effekt hos kunden.

At holde projektet indenfor scope er derfor en primær prioritet angående kvalitetskontrol.

\section{Plans}
\label{sec:plans}
Formålet med en plan er at identificere og tildele opgaver indenfor organisationen, hvor mange resourcer opgaverne skal tildeles, hvad deres tidshorizont er og deres fordele for projektet.

Udfra planen kan en baseline defineres af højere management, der anvendes af projektet project manager til at vurdere projektets fremgang indenfor planen. Det gør at eventuelle afvigelser kan identifiers og handles på så tidligt som muligt i planen.

Der eksistere flere typer planer, hvert med deres specifikke scope.

\subsection*{Project plan}
Project plan er den gennemløbende plan for projektets levetid. Den beskriver det store billede eksempelvis ved brug af projekt milestones.

\subsection*{Stage plan}
Stage plan ligner projct plan men eksiterer på et mere detaljeret plan i en begrænset tidsperiode af projektets levetid. I en stage plan detaljeres det opgaver til en grad, der tillader at det daglige arbejde kan valideres af projektets PM.

Da stage plans udarbejdes flere gange i projektet kan efterfølgende stage plans inkludere erfaringer fra tidligere planer.

\subsection*{Team plan}
Afhængig af teamets størrelse og organisationens krav kan en team manager udvikle en team plan. Heri kan team manageren mere konkret styre udarbejdelse af teamets arbejde internt. 

\subsection*{Exception plan}
En exception plan kan klargøres på forhånd i tilfælde af større afvigelser i én eller flere KPI'er. Formålet er at ét bestemt niveau af management hierarkiet kan dokumentere handlinger, der bør tages i forbindelse med afhjælpningen af afvigelsen.

Målet er at den aktuelle exception plan overtager enten den nuværende stage plan, eller den overordnede project plan, i tilfælde af større afvigelser. Derfor udvikles en exception plan indenfor det scope, hvori den er designet til at overtage.

\section{Risk}
Risk management under PRINCE2 løber i 3 stadier:

\subsection*{Identifikation}
Dokumentation af risks, der overvejes, hvorved der skabes en fælles forståelse for den enkelte risks påvirkning af projektet.

\subsection*{Vurdering}
Categorisering af risks under følgende kernevurderinger:
\begin{itemize}
    \item Sandsynlighed
    \item Alvorlighed af indvirkning
    \item Umiddelbarhed
\end{itemize}

\subsection*{Kontrol}
Identifikation af både potentielle handlinger til at mitigere indvirkningen og tovholdere, udførelse af identificerede handlinger og overvågning af udviklingen efterfølgende.

\section{Change}
Ændringer til projektet efter start kontrolleres vha. et issue register. I et issue register bliver ændringsforslag registreret og arkiveret. Det tillader projektets project manager at løbende holde øje med og opfylde ændringsforslag.

Ændringsforslag kategoriseres under følgende prioriteter, som løbende genvurderes:

\subsection*{Must have}
Essentielle ændringer, der er kritiske for projektets succes.

\subsection*{Should have}
Vigtige ændringer, der svækker det endelige resultat hvis ikke implementeret.

\subsection*{Could have}
Ikke-essentielle ændringer, der ikke svækker det endelige resultat hvis ikke implementeret.

\subsection*{Won't have}
Ændringen er ikke vigtig, og skal ikke implementeres uden genvurdering.

Under kategoriseringen skal fordele ved implementering vægtes imod den forventede omkostning for at give den endelige kategorisering.

\section{Progress}
Formålet med progress temaet er at etablere systemer/processer til at overvåge projektets fremgang. Det kan bruges til at vurdere den endelige tidshorizont for enten hele projektet eller en mindre iteration.

Hvert niveau af projekthierarkiet holder øje med bestemt sektion af fremgangen. Højere management tager sig af det overordnede fremgangsbillede, hvor lavere dele af hierarkiet fokusere på et kortere plan og rapporterer opad.

Til at vurdere fremgangen på de forskellige niveauer kombineres mange af de tidligere målepinde såsom planer, risks og ændringsforslag.
