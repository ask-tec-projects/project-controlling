\chapter{De 7 "Principles"}

\section{Continued Business Justification}
Der skal løbende igennem projektets udvikling levetid eksistere en business-orienteret begrundelse for projektets fortsættelse.

\section{Learn from Experience}
Erfaring fra tidligere projekter skal benyttes, og erfaringer nuværende projekter skal dokumenteres, så de kan benyttes i fremtiden. Det foregår i følgende perioder:

\subsection*{Projektstart}
Ved projektopstart skal tidligere erfaringer gennemgås for bedre at kunne planlægge det nye projekt og undgå problemer, der tidligere er blevet mødt. Tidligere erfaringer kan også bistå i udbedringen af udfordringer i løbet at projektet.
\subsection*{Projktudvikling}
Under projektudviklingen skal erfaringer dokumenteres så de kan anvendes i fremtidige projekter.
\subsection*{Projektafslutning}
Ved afslutningen af et projekt skal erfaringer arkiveres og kategoriseres efter om de er "learnings" (der er sket ændringer grundet erfaringen) eller "lessons" (ingen ændringer er foretaget, der er blot lavet en identifikation).

\section{Defined Roles and Responsibilities}
\label{sec:defined_roles_and_responsibilities}
Det er vigtigt at projektet har en veldefineret projekt management struktur. Det betyder at de primære stakeholders er kendt indenfor projektet.

\subsection*{Business}
Business stakeholders har anvar for at sikre at organisationens investering i projektet giver god værdi i forhold til investeringens størrelse.

\subsection*{Users}
User stakeholders agerer som den endelige kunde, og de er målet for projektets funktion.

\subsection*{Suppliers}
Supplier stakeholders forsyner projektet med resourcer og expertise. Kombineret sørger alle 3 stakeholder grupper for at holde projektet balanceret, og det udlægger den enkeltes anvsarsområde og opgaver.

\section{Manage by Stages}
PRINCE2 foreskriver 2 planer; en kortsigtet team plan og en langsigtet projektplan. Den kortsigtede plan beskriver de umiddelbare opgaver for teamet, mens den langsigtede plan giver det store perspektiv. Hvert stage evalueres efterfølgende og lessons/learnings tages til efterretning i efterfølgende stages.

\section{Manage by Exception}
Det er ikke effektivt at have højere mangement som ansvarlig for at styre projektets KPI'er direkte. Der ligges derfor et generelt udlæg som baseline, hvilket projekt manageren bruger til at korigere projektets kurs. Hyppigt anvendte KPI'er er:
\begin{itemize}
    \item Tid
    \item Omkostninger
    \item Kvalitet
    \item Scope
    \item Risiko
    \item Udbytte
\end{itemize}
Her er det en fordel at sætte kontroller op på KPI'erne for at sikre at de ikke afviger ukontrolleret fra baseline sat af højere management.

\section{Focus on Products}
Ifølge PRINCE2 er et succesfuldt projekt produktorienteret og ikke aktivitetsorienteret. Det vil sige at kvalitet og ønsker fra projektets stakeholders er i fokus. Heri er det vigtigt at holde projektets scope for øje for at undgå "scope creep".

\section{Tailor to suit the project environment}
PRINCE2 kan anvendes til et stort array af projekttyper, og det kan derfor skræddersyes til det aktuelle miljø. Formålet med adapteringen er at PM processerne stemmer overens med de eksisterende business processer og projektets/organisationens generelle miljø. Deri ligger også rapporteringen og vægtningen af projektets KPI'er.
